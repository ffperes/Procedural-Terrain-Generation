\documentclass{beamer}
\usepackage{pgfpages}
%\pgfpagesuselayout{resize to}[a5paper,border shrink=0mm,portrait]
%\pgfpagesuselayout{2 on 1}[a4paper,border shrink=5mm]
\usepackage{beamerthemesplit}
\usepackage{verbatim}
\usepackage{hyperlatex}
\input{people.tex}
\logo{\includegraphics[height=1cm]{sigma.png}}
\title{Linear Schemas for Program Dependence}
\author{Sebastian Danicic}
\date{\today}
%\setbeamercolor{title}{fg=red!80!black,bg=red!20!white}
%\usecolortheme{seahorse}
%\usecolortheme{rose}
%\usetheme{Warsaw}
%\usetheme{Frankfurt}
%\usetheme{AnnArbor}     
%\usetheme{Ilmenau} %nice
%\usetheme{Antibes} %nice     
%\usetheme{JuanLesPins}
%\usetheme{Bergen}  %nope!      
%\usetheme{Luebeck} %nope!
%\usetheme{Berkeley} %hmmm    
\usetheme{Madrid} %best so far
%\usetheme{Berlin}       
%\usetheme{Malmoe}
%\usetheme{Boadilla} %very clean     
%\usetheme{Marburg}
%\usetheme{boxes}  %same as madrid      
%\usetheme{Montpellier}
%\usetheme{CambridgeUS}  %very nice simple and red
%\usetheme{PaloAlto}
%\usetheme{Copenhagen}   
%\usetheme{Pittsburgh}
%\usetheme{Darmstadt}    
%\usetheme{Rochester}
%\usetheme{default}      
%\usetheme{Singapore}
%\usetheme{Dresden}      
%\usetheme{Szeged}
%\usetheme{Frankfurt}    
%\usetheme{Warsaw}
%\usetheme{Goettingen}   
%\usetheme{Hannover}

%\usefonttheme[onlylarge]{structuresmallcapsserif}

\begin{document}

\section{Introduction}

\frame{\titlepage}

%\section[Outline]{}
%\frame{\tableofcontents}


\begin{frame}
\frametitle{First Schema Meeting 21 Nov 2006}

Vague Agenda: \begin{tabular}{ll}
            10-10.30: &Coffee etc.\\
	    10.30-11.15: &Sebastian Danicic: Intro to Schemas Project\\
	    11.15-12.00:& Mike Laurence: A History of Schemas: Results and\\
	     &Open Problems\\
	    12.00-1.15: &Lunch (probably in College)\\
	    1.15-1.30: &``Is that the Post Office Tower?'' \\
	    &(Traditional Circuit of Goldsmiths)\\\\\\	
	    1.30-4:& Future Work - Discussion\\
	    4-6: &Drinking in the Hobgoblin Pub\\
	    6- & Eating in The Thailand, 15 Lewisham Way
	    \end{tabular}
\end{frame}



\begin{frame}
%\begin{center}
\frametitle{Linear Schemas for Program Dependence~ EP/E002919/1 
}
%Value:& {\pounds}301,452\\
\begin{center}
\uncover<1->{\includegraphics[height=1cm]{epsrc.jpg}}
\end{center}

\begin{center}
\uncover<2->{\includegraphics[height=1cm]{gold.jpg}}
\uncover<3->{\includegraphics[height=1cm]{kings.png}}
\uncover<4->{ \includegraphics[height=1cm]{brunel.png}}
\uncover<5->{\includegraphics[height=1cm]{essex.png}}
\uncover<6->{\includegraphics[height=1cm]{atuk.png}}
\uncover<7->{\includegraphics[height=1cm]{atandt.png}}


\end{center}
	
%\begin{semiverbatim}http://gow.epsrc.ac.uk/ViewGrant.aspx?GrantRef=EP/E002919/1\end{semiverbatim}\\
%\end{center}	
\end{frame}


\begin{frame}
%\begin{center}
\frametitle{This Talk}
%Value:& {\pounds}301,452\\


\uncover<1->{History of the Schemas Project}\\
\uncover<2->{Aims of the Schemas Project}\\
\uncover<3->{People in the Schemas Project }\\
\uncover<4->{Motivation - Examples from Program Slicing}\\
\uncover<5->{Semantics of Schemas and Dataflow Minimality}\\
\uncover<6->{Decidability of Equivalence of Schemas}\\
\uncover<7->{Future Work}
\end{frame}
\section{History of Schema Theory}
\begin{frame}
  \frametitle{History of Schema Theory}
 \begin{itemize}
 \item<1->{Schema theory was introduced in the 1950s by a Russian Mathematician, 
 Ianov. It was seen as a way of proving the correctness of compiler optimisations.}
 
 \item<2->{Schemas are an abstract way of representing classes of programs
 with identical structure.}

 
\end{itemize}
\end{frame}  

\begin{frame}
  \frametitle{Seminal Work on Program Schemas}

Some well-known computer scientists have worked on Schemas:
	\begin{itemize}
	\item<2->Ianov (1960)``The Logical Schemes of Algorithms'' 
	\item<2-> M.S. Paterson (1968) ``Program Schemata''
	\item<2->  D.C. Cooper(1969) ``Program Scheme Equivalences and Logic''
	\item<2-> R.Milner(1970) ``Equivalences on Program Schemes''
	\item<2->  Ershov (1971) ``Theory of Program Schemata''
	\item<2->  Constable and Gries(1972) ``On Classes of Program Schemata''
	\item<2->  Garland and Luckham(1973)``Program Schemes, Recursion Schemes and Formal
	Language''
	\item<2->  A.K.Chandra (1973) ``On the Properties and Applications of
	Program Schemas''
	
	\end{itemize}

\end{frame}  


\begin{frame}
 \frametitle{Death of Schemas}
 
\begin{center}
\includegraphics[height=2cm]{sad.jpg}\\
  {The subject more or less died out by the 1980s due to a lack of
positive results.}\\
\includegraphics[height=2cm]{rip.png}\\

\end{center}
\end{frame}  
 

\begin{frame}[fragile,t]
  \frametitle{History}

\uncover<3>{\includegraphics[height=2cm]{ghost.png}}\\
 \begin{itemize}
 \item<1->{ In the 1990s, Mark and I were working on the computability of Dataflow Minimal
 Slices.} 
 
 \item<2->{Mark was at a conference and mentioned our problem to Tom Reps of
 Wisconsin and  Tim Teitelbaum of Cornell who suggested we looked at 
 }\only<3>{\alert<3>{Schema Theory}.}
 \end{itemize}

\end{frame}

\begin{frame}
  \frametitle{They were right!}
\begin{itemize}

\item<2-5>{For our problem it turned out we needed a class of schema in which no
predicate or function symbol occurs more than once.} 

\item<3-5>{We called these \alert{Linear} Schemas.}
\item<4-5>{We were surprised that no work had been done on Linear Schemas.}
\item<5>{Serendipitously, it turned out  that the linearity condition helped in proving
decidability of equivalence of schemas. }
\end{itemize}
\end{frame}
	


\begin{frame}
  \frametitle{Mike's Work}

\begin{itemize}
\item<1->{
Mike proved that equivalence of conservative, free, linear
schemas is decidable.} 

\item<2->{ and in his thesis he strengthened this by proving that equivalence of
liberal, free linear schemas is decidable in polynomial time.}
\item<3->{
This 
represented significant progress in the field of schema theory after a hiatus of
about twenty years.} 


\end{itemize}
\end{frame}

\begin{frame}
But we still don't know whether dataflow minimal slices are
computable!\\\includegraphics[height=2cm]{sad.jpg}

\end{frame}

\section{Aims of the Schemas Project}
\begin{frame}
  \frametitle{Aims of the Schemas Project}

\begin{itemize}

\item<1->{
There is strong evidence that the imposition of this extra
but natural condition of linearity (or partial forms of linearity) will lead to
further decidability results in the theory of schemas.} 

\item<2->{We hope that
our new results will lead to a re-appraisal of the substantial body of work in
program schema theory and to further research on its applications in a modern
framework. \uncover<3->{\\e.g. Dataflow Minimal Slicing!}
}

\end{itemize}
\end{frame}

\section{People}


  \begin{frame}
\begin{center}
People 
\end{center}
 \end{frame}


%\subsubsection{Sebastian Danicic}

  \begin{frame}
 
\frametitle{\includegraphics[height=1.5cm]{seb.jpg} Dr. Sebastian Danicic (Principal Investigator)}
\begin{itemize}
\item<2->{Sebastian is a Lecturer in Computing at Goldsmiths.}
\item<3-> His work on slicing began in the mid 1990s with work on its semantic
foundations   and algorithms.  
\item<4->{The Schemas project described here has grown out of the research Sebastian
conducted during his PhD. Since then, he has supervised
two PhDs continuing this work.} 
\end{itemize}
 \end{frame}
	  

%\subsubsection{Mark Harman}

\begin{frame}
\frametitle{\includegraphics[height=1.5cm]{MarkHarman.png} Prof. Mark Harman(Co--Investigator)}
\begin{itemize}

\item<2->{Mark is Professor of Software Engineering at King's College London.}
\item<3->He is the principal investigator on a number of projects
including one
to design and build an industrial  program slicing and variable dependence analysis system.
 
\item<4->{He is the inventor of Amorphous Slicing} 
 
\item<5-> {He has also worked  on problems in software testing  
and upon the combination of slicing, transformation
and testing.}
\item<6->{Mark is also well--known for his work on Search-Based Software Engineering for which he 
has recently obtained a {\pounds}1,145,357 EPSRC grant.} 
\item<7->{Mark used to work at Goldsmiths}
\end{itemize}

\end{frame}


%\subsubsection{Rob Hierons}


\begin{frame}
\frametitle{\includegraphics[height=1.5cm]{rhierons.jpg}Prof. Rob Hierons (Co-Investigator)
}
\begin{itemize}

\item<2-> {Rob is Professor of Computer Science at Brunel University}.

\item <3->{Rob has a BA in Mathematics (Trinity College, Cambridge), and a Ph.D. in Computer Science (Brunel University)}

\item<4-> {Rob has published widely in the areas of testing and software
engineering.}

   
\item<5>{ Rob used to work at Goldsmiths}
\end{itemize}

\end{frame}


%\subsubsection{Rob Hierons}


\begin{frame}
\frametitle{\includegraphics[height=1.5cm]{fox.jpg}Dr. Chris Fox}
\begin{itemize}

\item<2->{Chris is Reader in Computer Science at Essex University.}

\item<3->{He has published widely in the areas of {Formal Semantics}, 
Philosophy of Language, Business Process Modelling
and Program Analysis.}

\item<4->{Chris implemented the first conditioned slicer, ConSiT: a system
which combines symbolic program execution with theorem proving.}

\item<5>{Chris used to work at Goldsmiths}

\end{itemize}

\end{frame}



\begin{frame}
\frametitle{\includegraphics[height=1.5cm]{lahcen.jpg}Dr. Lahcen Ouarbya}
\begin{itemize}
\item<1->{Lahcen is a lecturer at Goldsmiths College.}
\item<2->{He has a BSc in Physics from �Cadi Ayad Univesity,  Marrakech, Morocco }
\item<3->{...an MSc in Solid State Physics  from St. Petersburg State University }
\item<4->{and a PhD in Formal Semantics of Slicing  from Goldsmiths, (supervised by
Sebastian,  Mark, Rob, Chris  and John).}
\end{itemize}
\end{frame}



\begin{frame}
\frametitle{\includegraphics[height=1.5cm]{dave.jpg}Dr. Mohammed (Dave)
Daoudi\includegraphics[height=1.5cm]{arab.png}}
\begin{itemize}
\item<1->{Dave is now studying Financial Mathematics at Imperial College.}

\item<2->{Dave has a first class honours degree in Mathematics from Goldsmiths}


\item<3->{ and a PhD in Conditioned Slicing  
from Goldsmiths, (supervised by
Sebastian,  Mark, Rob, Chris  and John).}

\end{itemize}
\end{frame}



\begin{frame}
\frametitle{\includegraphics[height=1.5cm]{mike.jpg}Dr. Dr. Mike Laurence (Research Assistant)}
\begin{itemize}
\item<1->{Mike has {BSc in Mathematics from Queen Mary College, University of London (1st class honours)}
an has an {MSc in Mathematics from University of Warwick}}
\item<2->{A PhD in Group Theory from Queen Mary College} 
\item<3->{A PhD in Schema Theory  
from Goldsmiths, (supervised by
Sebastian, Rob, Mark and John).}

\item<4->{He has had a number of previous jobs notably as a {clothes
sorter at an Oxfam shop near Goodge Street}}.
\end{itemize}
\end{frame}
    



\begin{frame}
\frametitle{\includegraphics[height=1.5cm]{mike.jpg}Dr. Dr. Mike Laurence (Research Assistant)}
\begin{center}
His favourite food is:
\\\uncover<2->{\includegraphics[height=5cm]{Pistachio.jpg}\\Pistachio Nuts!}
\end{center}
\end{frame}

\begin{frame}
\frametitle{\includegraphics[height=1.5cm]{john.jpg}Dr. John Howroyd (Industrial Collaborator)}
\begin{itemize}
\item<1->{John works as Director of Research and Development at @UK PLC.}
\item<2->{He has a BSc in Mathematics from Oxford University and a PhD in Mathematics from St. Andrews.}
\item<3->{His interests include Geometric Measure Theory, Harmonic Analysis,
Isometries, Applications of Domain Theory in Computable Mathematics, and Static
Program Analysis.}
\item<4->{John used to work at Goldsmiths}
\end{itemize}
\end{frame}


\begin{frame}
\frametitle{\includegraphics[height=1.5cm]{weyuker.jpg}Prof. Elaine Weyuker (Industrial Collaborator)}
\begin{itemize}
\item<1->{Elaine works at AT\&T Labs-–Research.}
\item<2->{She has published widely in the area of schema theory and dataflow
analysis.}
\item<3->{She   is the originator of dataflow testing.}
\item<4->{She is world renowned for her
research into providing formal bases for  many areas of computer science.}
\item<5->{Elaine isn't here :(}
\item<6->{and ...}
\item<7->{Elaine has no connection with Goldsmiths}
\end{itemize}
\end{frame}
% 


\begin{frame}
\frametitle{A few of years ago...}
  \begin{center}
\includegraphics[height=7cm]{vastt-kcl.jpg}
\end{center}
\end{frame}

\begin{frame}
\frametitle{A few of years ago...}
  \begin{center}
\includegraphics[height=7cm]{group-ira.jpg}
\end{center}
\end{frame}

\begin{frame}
\frametitle{A few of years ago...}
  \begin{center}
\includegraphics[height=7cm]{group2p.jpg}
\end{center}
\end{frame}

\begin{frame}
\frametitle{A few of years ago...}
  \begin{center}
\includegraphics[height=7cm]{megroup.jpg}
\end{center}
\end{frame}


\begin{frame}
\frametitle{A few of years ago...}
  \begin{center}
\includegraphics[height=7cm]{groupOct01.jpg}
\end{center}
\end{frame}


\begin{frame}
\frametitle{A few of years ago...}
  \begin{center}
\includegraphics[height=7cm]{sgroup-dec3-2001.jpg}
\end{center}
\end{frame}


\begin{frame}
\frametitle{A few of years ago...}
  \begin{center}
\includegraphics[height=7cm]{vdec2002.jpg}
\end{center}
\end{frame}

\section{Motivation - Examples from Slicing}

\begin{frame}
\frametitle{Background -- Program Slicing}
\begin{center}
\includegraphics[height=4cm]{prism.png}\\
\uncover<2->{Program Slicing gives us different views of the same program\\}
\uncover<3>{...depending what we are interested in.}
\end{center}
\end{frame}



\begin{frame}
\frametitle{Questions answered by Slicing}
\begin{itemize}
\item<1-> Which bits of big program $P$ affect the final value of variable $x$?
\item<2-> Which bits of big program $P$ affect the updating of this file?
\item<3-> Which bits of big program $P$ affect the firing of this missile?

\end{itemize}
\end{frame}



\begin{frame}
\frametitle{Commercial Slicing Tools: Kaveri/Indus}
\begin{center}
\includegraphics[height=1cm]{banner.png}\\

\includegraphics[height=3cm]{slice.png}\\

Kaveri is an eclipse plug-in front-end for the Indus Java slicer. It utilizes
the Indus program slicer to calculate slices of Java programs and then displays
the results visually in the editor. The purpose of this project is to create an
effective tool for simplifying program understanding,  program analysis, program
debugging and testing.

\end{center} 

\end{frame}




\begin{frame}
\frametitle{Commercial Slicing Tools: Codesurfer}
\begin{center}
\includegraphics[height=5cm]{BackwardSlice.png}\\

{\it``The backward slice from a program point p includes all points that
\alert<2->{may} influence
whether control reaches p, and all points that \alert<2->{may} influence the values of the
variables used at p when control gets there.'' 
}\\
\uncover<2>{What the hell does \alert<2>{may} mean?}
\end{center}
\end{frame}



\begin{frame}
\frametitle{The Crux of the Problem}
\begin{itemize}
\item<1-> Slicing algorithms are conservative: They often keep in
bits of the program the could be left out.

\item<2-> A statement that a slicing algorithm thinks may affect a variable
often does not!

\item<3-> This leads to slices that are too big. 

\item<4->Small is beautiful. -- Big slices aren't very useful.

\item<5-> We want to find ways of producing smaller slices.
\end{itemize}
\end{frame}

\section{Examples}

\begin{frame}
\frametitle{Examples}
We now give two examples showing wrong computation of dependence
by slicing algorithms:
\begin{itemize}
\item<2-> The ``c becomes one'' example
\item<3-> The ``Montreal Boat Trip'' example (John Howroyd)
\end{itemize}
\end{frame}


\begin{frame}[fragile,t]

  \frametitle{The ``c becomes one'' example}

Which lines of this program affect the final value of z?
\begin{center}
\begin{semiverbatim}

     \alert<2-3,6->{while (i<k)\uncover<6,7>{\alert<6,7>{  <-----}}
     \alert<2-3,6-> \{}
         \alert<2->{if (c<5)} \uncover<4,5,6>{\alert<4,5,6>{  <-----}}
         \alert<2->{\{}
             \alert<2->{z=7;} \uncover<4>{\alert<4>{  <-----}} 
             \alert<2-3,5,6->{c=y+c;}\uncover<5>{\alert<5>{ <-----}}
         \alert<2->{\}}
         \alert<2-3,7->{i=i+1;}\uncover<7>{\alert<7>{ <-----}}
      \alert<2-3,6->{\}}
\end{semiverbatim}
\end{center}
\only<1>{}
\only<2>{Conventional Program Slicers like Codesurfer will say ``all of them!''}
\only<3>{but why?}
\only<4>{\alert{z=7} is {\em control--dependent} on \alert{(c$<$5)}}
\only<5>{Because its in a loop \alert{c$<$5} is data-dependent upon \alert{c=y+c;}}
\only<6>{The \alert{if} is control--dependent the guard of the \alert{while}}
\only<7>{The guard of the \alert{while} is data-dependent on \alert{i=i+1}}
\only<8>{So slicing on z gives the whole program}
\only<9>{Slicing Algorithms compute the transitive closure of the dependence relation}
\end{frame}







\begin{frame}[fragile,t]

  \frametitle{The ``c becomes one'' example}

Which lines of this program affect the final value of z?
\begin{semiverbatim}

     \alert<1->{while (i<k)
     \alert<1-> \{}
         \alert<1->{if (c<5)}
         \alert<1->{\{}
             \alert<1->{z=7;}
             \alert<1>{c=y+c;} {<-----}
         \alert<1->{\}}
         \alert<1->{i=i+1;}
      \alert<1->{\}}
\end{semiverbatim}
\only<1>{But is z really dependent on this?}
\only<2>{Clearly not because if we do execute c=y+c the value of z can't change any further,
so it is irrelevant if we go through the true part of the if after that.}
\only<3>{So Transitive closure of dependence doesn't seem to be the most accurate was of computing dependencies.}
\only<4>{As we have seen, dependence is not transitive.}
\end{frame}



\begin{frame}[fragile,t]

\frametitle{Conventional Slicing Removes Non--termination}


\begin{semiverbatim}

     while (true)
     \{

     \}
     \alert<2-5>{z=2;}
\end{semiverbatim}
\only<1>{What do we get if we slice on the final value of z?}
\only<2>{The loop is removed since z=2 is not data or control dependent on it.}
\only<3>{So transitive closure of dependence can introduce termination.}
\only<4->{So, formally a program $p$  and its slice $s$ need only agree in initial states where $p$ terminates.}
\only<5->{\\So, there's an even smaller slice of this program.}
\only<6->{\\The empty program -- all statements can be removed.}
\end{frame}


\begin{frame}
\frametitle{The Semantics of End--Slicing}
Definition:



\begin{center}


$q$ is an \alert{end slice} of $p$ withe respect to variable $v$\\
   if and only if\\
for all initial states $\sigma$\\
${\mathcal M}[\vert p \vert]\sigma \neq \bot \implies {\mathcal M}[\vert p
\vert]\sigma v = {\mathcal M}[\vert q \vert]\sigma v$
\mbox{}\\
\mbox{}\\

\begin{itemize}
\item<2-> Slicing is reflexive.
\item<3-> Slicing is transitive.
\item<4-> Slicing is not symmetric.	
\item<5-> Slicing is not anti-symmetric.	
\end{itemize}

\end{center}

\end{frame}

\begin{frame}[fragile,t]
\frametitle{The ``Montreal Boat Trip'' example}
  \begin{center}
\includegraphics[height=6cm]{montrealboat.jpg}\\	
\only<1>{Written on the white board by me at SCAM 2002 in Montreal!}
\only<2>{I asked, `` What is the slice on variable j''}
\only<3>{To which Ira Baxter wittily replied:}
\end{center}
\end{frame}





\begin{frame}[fragile,t]
\frametitle{The ``Montreal Boat Trip'' example}
  \begin{center}
\includegraphics[height=6cm]{fat.jpg}\\	
{\large Who cares!}
\end{center}
\end{frame}


\begin{frame}[fragile,t]

\frametitle{The ``Montreal Boat Trip'' example}

\only<1-6>{What is the slice on $j$ at the end of the program?}
\only<7->{Linear Schemas}


\begin{semiverbatim}
     \alert<2->{while p(j)
     \alert<2-> \{}
         \alert<2->{if q(k) k=f(k);}
         \alert<2->{else}
         \alert<2->{\{}
             \only<1-8>{\alert<2->{k=g(k);}}\only<9>{k=f(k);} \uncover<3-6>{  <-----}
             \alert<2->{j=h(j);}
         \alert<2->{\}}
      \alert<2->{\}}
\end{semiverbatim}
\only<2>{Again, transitive closure of dependence gives the whole program.}
\only<3>{But what about this line?}
\only<4>{It either causes the program to non--terminate or
increases the number of iterations of the loop before termination.}
\only<5>{In initial states where the program terminates this line doesn't affect the final value of j.}
\only<6>{So k=g(k) can be removed from the slice.}
\only<7>{The program above is in fact a \alert{Schema}}
\only<8>{it is in fact a {\em linear} Schema -because each function and
predicate symbol
occurs at most once} 
\only<9>{Now it's not {\em linear} }
\end{frame}

\section{Dataflow Minimality and Semantics}
\begin{frame}
\frametitle{The Data Flow Minimality Problem:}

\begin{itemize}

\item<1->{ We have shown that transitive closure of dependence gives over--large
slices.}


\item<2->{Small slices are good.}




\item<3->{So is it possible to produce \alert{minimal slices} at this level 
of abstraction?}	

\item<4->{A \alert{minimal slice} is a slice all of whose proper sub--programs
are not slices.}	


\end{itemize}
\end{frame}

	

\begin{frame}[fragile,t]

\frametitle{The Semantics of Schemas}

\uncover<2->{States = [Variables $\rightarrow$ Terms] $\cup \{\bot\}$\\}
\uncover<3->{\only<3>{(Herbrand)} Interpretations = [Terms $\rightarrow$	 \{True,False\}]\\}
\uncover<5->{${\mathcal M}$: Schemas $\rightarrow$ Interpretations $\rightarrow$ States
$\rightarrow$ States.}			  

\begin{semiverbatim}
\uncover<6->{ while p(j) 
     \{
        if q(k) k=f(k);
        else
         \{
             k=g(k);
             j=h(j);
         \}
     \}}
\end{semiverbatim}

\uncover<6->{Show John's Haskell Schema Interpreter readSch "boat.sch"}
\end{frame}




\section{Decidability of Equivalence}
\begin{frame}[fragile,t]
\frametitle{Decidability of Equivalence of Schemas}

\begin{itemize}



\item<1->{Two Schemas are equivalent if the are semantically equivalent
under all Herbrand interpretations.}



\item<2->\alert{{ 
The Decidability of Equivalence of Schemas implies the computability
of minimal slices.}}


\item<3->{Why?}

\item<4->{Crudely, we can simply try all combinations of deleting statements
and check for equivalence of the resulting schema with the original.}

\end{itemize}
\end{frame}


\begin{frame}[fragile,t]
\frametitle{\only<1>{The Bad News}\only<2>{Not So Bad News}\only<3>{The Good News}}
\only<1>{Paterson (1967):}
\only<3>{Mike Laurence (2006)}
%{Mike Laurence :}

\only<1>{\includegraphics[height=2cm]{sad.jpg}}
\only<3>{\includegraphics[height=2cm]{smiley.png}}


\only<1>{Equivalence of Schemas is Undecidable.}	
\only<2>{For Linear Schemas decidability of equivalence is an open problem.}	
\only<3>{For certain classes of Linear Schemas equivalence is decidable.}	




\end{frame}



\section{Future Work}

\begin{frame}
\frametitle{Open Problems in Decidability to Investigate}

\begin{center}
\begin{itemize}
\item <1->{Is equivalence of free  conservative linear schemas decidable?}
\uncover<2->{\alert{{Yes}}} 
\item <3->{Is equivalence of free liberal linear schemas decidable?}
\uncover<4->{\alert{Yes}} 
\item <5->{Is equivalence of free ~~~~~~~~ linear schemas decidable?}
\uncover<6->{\alert{Don't know}} 
\item <7->{Is equivalence of ~~~~ ~~~~~~~~ linear schemas decidable?}
\uncover<8->{\alert{Don't know}} 
\item <9->{Is freeness of linear schemas decidable?}
\uncover<10->{\alert{Don't know}} 
\end{itemize}
\end{center}
\end{frame}


\begin{frame}
\frametitle{More Accurate Notions of Dependency}
\begin{center}
\begin{itemize}
\item <1->{A new lazy trace semantics for Schemas for general Slicing}
\item <2->{Characterisation of Slicing in terms of this new Semantics}
\item<3->{Investigation of Decidability of Equivalence in terms of the new
Semantics}
\item <4->{Extending the Syntax and Semantics of Linear Schemas to handle
more features e.g. functions and procedures.}
\item <5->{New algorithms for computing dependency}
\end{itemize}
\end{center}

\end{frame}


\begin{frame}
\frametitle{Plan for the Schemas Project}
  \begin{center}
\includegraphics[height=8cm]{all.png}
\end{center}
\end{frame}


\begin{frame}[fragile,t]
\frametitle{{Publications so Far} }

%{Mike Laurence :}
\small
\begin{itemize}
\item<1->{Michael~R. Laurence, Sebastian Danicic, Mark Harman, Rob Hierons, and John
  Howroyd.\\
Equivalence of conservative, free, linear program schemas is
  decidable.\\
	{\em Theoretical Computer Science}, 290:831--862, January 2003.
}



\item<2->{Sebastian Danicic, Chris Fox, Mark Harman, Robert~Mark Hierons, John Howroyd,
  and Mike Laurence.
\\ Slicing algorithms are minimal for programs which can be expressed as
  linear, free, liberal schemas.
\\ {\em The Computer Journal}, 48(6):737--748, 2005.}

\item<3->{{Sebastian Danicic, Mark Harman, Robert~Mark Hierons, John Howroyd, and Mike
  Laurence.\\
Equivalence of linear, free, liberal, structured program schemas is
  decidable in polynomial time.\\
{\em Theoretical Computer Science}, 2006.}}


\end{itemize}

\end{frame}


\begin{frame}
\frametitle{Mike -- Over to you!}
  \begin{center}
\includegraphics[height=8cm]{mike.jpg}
\end{center}
\end{frame}



\end{document}


\begin{frame}[t,fragile]
  \frametitle{}

\end{frame}


\begin{frame}[fragile]
  \frametitle{An Algorithm For Finding Primes Numbers.}
\begin{semiverbatim}
\uncover<1->{\alert<0>{int main (void)}}
\uncover<1->{\alert<0>{\{}}
\uncover<1->{\alert<1>{ \alert<4>{std::}vector<bool> is_prime (100, true);}}
\uncover<1->{\alert<1>{ for (int i = 2; i < 100; i++)}}
\uncover<2->{\alert<2>{    if (is_prime[i])}}
\uncover<2->{\alert<0>{      \{}}
\uncover<3->{\alert<3>{        \alert<4>{std::}cout << i << " ";}}
\uncover<3->{\alert<3>{        for (int j = i; j < 100;}}
\uncover<3->{\alert<3>{             is_prime [j] = false, j+=i);}}
\uncover<2->{\alert<0>{      \}}}
\uncover<1->{\alert<0>{ return 0;}}
\uncover<1->{\alert<0>{\}}}
\end{semiverbatim}
  \visible<4->{Note the use of \alert{\texttt{std::}}.}
\end{frame}

\begin{frame}[fragile]

  \frametitle{The "c becomes one" example}

Which lines of this program affect the final value of z?
\begin{semiverbatim}

     \alert<4>{while (i<k)
     \alert<4> \{}
         \alert<3>{if (c<5)}
         \alert<3>{\{}
             \alert<2>{z=7;}
             \alert<6>{c=y+c;}\only<6>{ what about this statement?}\only<7>{ do we really need it?}
         \alert<3>{\}}
         \alert<5>{i=i+1;}
      \alert<4>{\}}
\end{semiverbatim}


\end{frame}



\begin{frame}
  \frametitle{}
  \begin{center}
\end{center}
\end{frame}


\begin{frame}[fragile,t]

  \frametitle{The ``c becomes one'' example}

Which lines of this program affect the final value of z?
\begin{semiverbatim}

     \alert<1->{while (i<k)
     \alert<1-> \{}
         \alert<1->{if (c<5)}
         \alert<1->{\{}
             \alert<1->{z=7;}
             {c=y+c;} So, yes We can leave it out.
         \alert<1->{\}}
         \alert<1->{i=i+1;}
      \alert<1->{\}}
\end{semiverbatim}
\end{frame}
